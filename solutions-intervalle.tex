\section*{Chapitre \ref{chap:intervalle}}
\addcontentsline{toc}{section}{Chapitre \protect\ref{chap:intervalle}}

\begin{solution}{5.1}
\begin{enumerate}
\item La fonction génératrice des moments d'une distribution Gamma avec paramètres $\alpha = 2$ et $\beta > 0$ est donnée par
$$
M_X(t) = (1-\beta t)^{-2}, \quad t < 1/\beta.
$$
La fonction génératrice des moments de $T = 2(X_1 + \cdots + X_n)/\beta$ est donc
$$
M_T(t) = \ex (e^{(2t/\beta) \sum_{i=1}^n X_i}) = \prod_{i=1}^n \ex \left\{ e^{(2t/\beta) X_i} \right\} = \{ M_X(2t/\beta)\}^n = (1-2t)^{-2n}
$$
à condition que $2t/\beta < 1/\beta$, i.e., si $t < 1/2$. Sachant que $(1-2t)^{-2n}$ est la fonction génératrice des moments d'une distribution khi-carrée avec $4n$ degrés de liberté, $T \sim \chi^2_{(4n)}$.

Ainsi, $T$ est une fonction de l'échantillon aléatoire et du paramètre $\beta$, avec une distribution connue qui ne dépend d'aucun paramètre inconnu~: $T$ est donc un pivot.

\item Pour construire l'intervalle de confiance bilatéral à partir de $T$, on note $\chi^2_{0,975, \, 4n}$ et $\chi^2_{0,025, \, 4n}$ les quantiles $2,5$\% et $97,5$\% d'une distribution $\chi^2$ avec $4n$ degrés de liberté. On a donc
$$
\Pr( \chi^2_{0,975, \, 4n} \le T \le \chi^2_{0,025, \, 4n}) = 0,95.
$$
On résout les inégalités
$$
 \chi^2_{0,975, \, 4n} \le T \le \chi^2_{0,025, \, 4n}
 $$
ou encore
 $$
 \chi^2_{0,975, \, 4n} \le \frac{2}{\beta} \sum_{i=1}^n X_i \le \chi^2_{0,025, \, 4n},
$$
ce qui donne un l'intervalle de confiance bilatéral de niveau $95$~\% pour $\beta$~:
\begin{align*}
\left[ \frac{2n \bar X_n}{ \chi^2_{0,025, \, 4n}} \, , \frac{2n \bar X_n}{ \chi^2_{0,975, \, 4n}}\right].
\end{align*}

\item Avec $n = 5$, les quantiles requis d'une distribution khi-carrée avec $4n = 20$ degrés de liberté sont
$$
 \chi^2_{0,975, \, 20} = 9.59, \quad  \chi^2_{0,025, \, 20} = 34.17.
$$
Sachant que $\bar x_n = 5,6$, l'intervalle de confiance de niveau $95$~\% pour $\beta$ est
\begin{align*}
\left[ \frac{2\times 5\times 5,6}{ 34.17} \, , \frac{2\times 5\times 5,6}{ 9.59}\right] = [ 1.64, 5.84].
\end{align*}
\end{enumerate}
	
\end{solution}
\begin{solution}{5.2}
    On cherche deux statistiques $L$ et $U$ tel que
    \begin{equation*}
      \prob{L \leq \mu \leq U} = 1 - \alpha.
    \end{equation*}
    On sait que si $X_1, \dots, X_n$ est un échantillon aléatoire tiré
    d'une distribution $N(\mu, \sigma^2)$, alors $\bar{X} \sim N(\mu,
    \sigma^2/n)$ ou, de manière équivalente, que
    \begin{displaymath}
      \frac{\bar{X} - \mu}{\sigma / \sqrt{n}} \sim N(0, 1).
    \end{displaymath}
    Par conséquent,
    \begin{equation*}
      \Prob{-z_{\alpha/2} \leq \frac{\bar{X} - \mu}{\alpha/\sqrt{n}} \leq
        z_{\alpha/2}} = 1 - \alpha,
    \end{equation*}
    d'où
    \begin{equation*}
      \Prob{\bar{X} - \frac{\sigma}{\sqrt{n}}\, z_{\alpha/2} \leq \mu
        \leq \bar{X} + \frac{\sigma}{\sqrt{n}}\, z_{\alpha/2}} = 1 - \alpha.
    \end{equation*}
    Les statistiques $L$ et $U$ sont dès lors connues: $L = \bar{X} -
    \sigma z_{\alpha/2}/\sqrt{n}$ et $U = \bar{X} + \sigma
    z_{\alpha/2}/\sqrt{n}$. Un estimateur par intervalle de $\mu$ est
    donc
    \begin{equation*}
      (\bar{X} - \sigma z_{\alpha/2}/\sqrt{n}, \,
      \bar{X} + \sigma z_{\alpha/2}/\sqrt{n}).
    \end{equation*}
    Avec $n = 20$, $\sigma^2 = 80$ et $\bar{x} = 81,2$, on obtient
    l'intervalle $(77,28, \, 85,12)$.
  
\end{solution}
\begin{solution}{5.3}
    On a $X \sim N(\mu, 9)$. Tel que démontré à
    l'exercice~\ref{chap:intervalle}.\ref{ex:intervalle:moyenne},
    \begin{equation*}
      \Prob{\bar{X} - \frac{\sigma}{\sqrt{n}}\, z_{0,05} \leq \mu
        \leq \bar{X} + \frac{\sigma}{\sqrt{n}}\, z_{0,05}} = 0,90.
    \end{equation*}
    Pour satisfaire la relation $\prob{\bar{X} - 1 < \mu < \bar{X} +
      1} = 0,90$, on doit donc choisir
    \begin{equation*}
      \frac{\sigma}{\sqrt{n}}\, z_{0,05} =
      \frac{3 (1,645)}{\sqrt{n}} = 1.
    \end{equation*}
    On trouve que $n = 24,35$. On doit donc choisir une taille
    d'échantillon de $24$ ou $25$.
  
\end{solution}
\begin{solution}{5.4}
    On sait que
    \begin{align*}
      \frac{\bar{X} - \mu}{S/\sqrt{n}} &\sim t(n-1) \\
      \intertext{et que}
      \frac{(n-1) S^2}{\sigma^2} &\sim \chi^2(n - 1).
    \end{align*}
    Ainsi, on peut établir que
    \begin{equation*}
      \Prob{\bar{X} - \frac{S}{\sqrt{n}}\, t_{\alpha/2} \leq \mu
        \leq \bar{X} + \frac{S}{\sqrt{n}}\, t_{\alpha/2}} = 1 - \alpha.
    \end{equation*}
    et qu'un intervalle de confiance de niveau $1 - \alpha$ pour $\mu$
    est
    \begin{equation*}
      (\bar{X} - S t_{\alpha/2}/\sqrt{n}, \,
      \bar{X} + S t_{\alpha/2}/\sqrt{n}).
    \end{equation*}
    Avec $n = 17$, $\bar{x} = 4,7$, $s^2 = 5,76$ et $\alpha = 0,10$,
    on trouve que $\mu \in (3,7, \, 5,7)$.

    Pour la variance, on cherche des valeurs $a$ et $b$, $a \leq b$
    tel que
    \begin{equation*}
      \Prob{a \leq \frac{(n-1)S^2}{\sigma^2} \leq b} =
      \Prob{\frac{(n-1)S^2}{b} \leq \sigma^2 \leq \frac{(n-1)S^2}{a}} =
      1 - \alpha.
    \end{equation*}
    Plusieurs valeurs de $a$ et $b$ satisfont cette relation. Le choix
    le plus simple est $a = \chi_{1 - \alpha/2}^2(n - 1)$ et $b =
    \chi_{\alpha/2}^2(n - 1)$. Ainsi, un intervalle de confiance de
    niveau $1 - \alpha$ pour $\sigma^2$ est
    \begin{equation*}
      \left(
        \frac{(n-1)S^2}{\chi_{\alpha/2}^2(n - 1)}, \,
        \frac{(n-1)S^2}{\chi_{1 - \alpha/2}^2(n - 1)}
      \right).
    \end{equation*}
    Dans une table de la loi khi carré, on trouve que
    $\chi_{0,05}^2(16) = 7,96$ et que $\chi_{0,95}^2(16) = 26,30$,
    d'où $\sigma^2 \in (3,50, \, 11,58)$.
  
\end{solution}
\begin{solution}{5.5}
    On représente la taille des étudiantes en actuariat par la
    variable aléatoire $X$ et celle des étudiantes en génie civil par
    $Y$. On a
    \begin{align*}
      \bar{X} &\sim N(\mu_1, \sigma_1^2/15), &
      \bar{Y} &\sim N(\mu_2, \sigma_2^2/20), \\
      \frac{14 S_1^2}{\sigma_1^2} &\sim \chi^2(14), &
      \frac{19 S_2^2}{\sigma_2^2} &\sim \chi^2(19)
    \end{align*}
    et les valeurs des statistiques pour les deux échantillons sont
    \begin{align*}
      \bar{x} &= 152, & \bar{y} &= 154, \\
      s_1^2   &= 101, &  s_2^2  &= 112.
    \end{align*}
    \begin{enumerate}
    \item Si l'on suppose que $\sigma_1^2 = \sigma_2^2 = 81$, alors
      $\bar{X} \sim N(\mu_1, 5,4)$ et $\bar{Y} \sim N(\mu_2, 4,05)$.
      Par conséquent,
      \begin{gather*}
        \Prob{-1,645
          < \frac{\bar{X} - \mu_1}{\sqrt{5,4}} <
          1,645} = 0,90 \\
        \intertext{et}
        \Prob{-1,645
          < \frac{\bar{Y} - \mu_2}{\sqrt{4,05}} <
          1,645
        } = 0,90
      \end{gather*}
      d'où
      \begin{equation*}
        152 - 1,645 \sqrt{5,4} < \mu_1 < 152 + 1,645 \sqrt{5,4},
      \end{equation*}
      soit $148,18 < \mu_1 < 155,82$ et
      \begin{equation*}
        154 - 1,645 \sqrt{4,05} < \mu_2 < 154 + 1,645 \sqrt{4,05},
      \end{equation*}
      soit $150,69 < \mu_2 < 157,31$.
    \item Si la variance est inconnue, on a plutôt que
      \begin{equation*}
        \frac{\bar{X} - \mu_1}{S_1/\sqrt{15}} \sim t(14)
        \quad \text{et} \quad
        \frac{\bar{Y} - \mu_2}{S_2/\sqrt{20}} \sim t(19).
      \end{equation*}
      Or, $t_{0,05}(14) = 1,761$ et $t_{0,05}(19) =
      1,729$, d'où
      \begin{equation*}
        152 - 1,761 \sqrt{\frac{101}{15}} < \mu_1 < 152 + 1,761
        \sqrt{\frac{101}{15}},
      \end{equation*}
      soit $147,43 < \mu_1 < 156,57$ et
      \begin{equation*}
        154 - 1,729 \sqrt{\frac{112}{20}} < \mu_2 <
        154 + 1,729 \sqrt{\frac{112}{20}},
      \end{equation*}
      soit $149,91 < \mu_2 < 158,09$.
    \item On cherche un intervalle de confiance à $90~\%$ pour la
      différence $\mu_1 - \mu_2$. Si $0$ appartient à l'intervalle, on
      pourra dire que la différence entre les deux moyennes n'est pas
      significative à $90~\%$. Pour les besoins de la cause, on va
      supposer ici que $\sigma_1^2 = \sigma_2^2 = \sigma^2$. Or,
      puisque
      \begin{gather*}
        \frac{(\bar{X} - \bar{Y}) - (\mu_1 - \mu_2)}{%
          \sigma \sqrt{\frac{1}{15} + \frac{1}{20}}}
        \sim N(0, 1) \\
        \intertext{et que}
        \frac{14 S_1^2}{\sigma^2} + \frac{19 S_2^2}{\sigma^2}
        \sim \chi^2(33)
      \end{gather*}
      on établit que
      \begin{displaymath}
        \frac{(\bar{X} - \bar{Y}) - (\mu_1 - \mu_2)}{\sqrt{\frac{14
              S_1^2 + 19 S_2^2}{33} \left( \frac{1}{15} + \frac{1}{20}
            \right)}} \sim t(33).
      \end{displaymath}
      De plus, $t_{0,05}(33) \approx z_{0,05} = 1,645$, d'où
      l'intervalle de confiance à 90~\% pour $\mu_1 - \mu_2$ est
      \begin{displaymath}
        \mu_1 - \mu_2 \in -2 \pm 5,82.
      \end{displaymath}
      La différence de taille moyenne entre les deux groupes
      d'étudiantes n'est donc pas significative.
    \item Tel que mentionné précédemment,
      \begin{displaymath}
        Y = \frac{14 S_1^2}{\sigma_1^2} \sim \chi^2(14).
      \end{displaymath}
      Or, on trouve dans une table de la loi khi carré (ou avec la
      fonction \texttt{qchisq} dans \textsf{R}) que
      \begin{displaymath}
        \Prob{6,57 < Y < 23,68} = 0,90.
      \end{displaymath}
      Par conséquent,
      \begin{gather*}
        \Prob{
          6,57 < \frac{14 S_1^2}{\sigma_1^2} < 23,68
        } = 0,90 \\
        \intertext{ou, de manière équivalente,}
        \Prob{
          \frac{14 S_1^2}{23,68} < \sigma_1^2 < \frac{14 S_1^2}{6,57}
        } = 0,90.
      \end{gather*}
      Puisque $s_1^2 = 101$ dans cet exemple, un intervalle de
      confiance à $90~\%$ pour $\sigma_1^2$ est $(59,71, \, 215,22)$.
    \item Un peu comme en c), on détermine un intervalle de confiance
      pour le ratio $\sigma_2^2/\sigma_1^2$ et on conclut que la
      différence entre la variance des étudiantes en génie civil n'est
      pas significativement plus grande que celle des étudiantes en
      actuariat si cet intervalle contient la valeur $1$. À la suite
      des conclusions en c), il est raisonnable de supposer que les
      moyennes des deux populations sont identiques, soit $\mu_1 =
      \mu_2 = \mu$. On a que
      \begin{displaymath}
        F = \frac{S_1^2/\sigma_1^2}{S_2^2/\sigma_2^2}
        \sim F(14, 19).
      \end{displaymath}
      On trouve dans une table de la loi $F$ (ou avec la fonction
      \texttt{qf} dans \textsf{R}) que
      \begin{displaymath}
        \Prob{0,417 < F < 2,256} = 0,90.
      \end{displaymath}
      Par conséquent,
      \begin{displaymath}
        \Prob{
          \frac{0,417 S_2^2}{ S_1^2}
          < \frac{\sigma_2^2}{\sigma_1^2} <
          \frac{2,256 S_2^2}{ S_1^2}
        } = 0,90
      \end{displaymath}
      et un intervalle de confiance à $90~\%$ pour
      $\sigma_2^2/\sigma_1^2$ est $(0,462, \, 2,502)$. La variance
      $\sigma_2^2$ n'est donc pas significativement plus grande que
      $\sigma_1^2$.
    \end{enumerate}
  
\end{solution}
\begin{solution}{5.6}
    On sait que
    \begin{displaymath}
      \frac{(n-1)S^2}{\sigma^2} \sim \chi^2(n-1).
    \end{displaymath}
    Ainsi, pour des constantes $a$ et $b$, $a \leq b$, on a
    \begin{equation*}
      \Prob{a \leq \frac{(n-1)S^2}{\sigma^2} \leq b} =
      \Prob{\sqrt{\frac{(n-1)S^2}{b}} \leq \sigma \leq \sqrt{\frac{(n-1)S^2}{a}}} =
      1 - \alpha.
    \end{equation*}
    Un estimateur par intervalle de $\sigma$ est donc $(\sqrt{(n-1)
      S^2/b}, \sqrt{(n-1) S^2/a})$, où $a$ et $b$ satisfont la relation
    $\prob{a \leq Y \leq b} = 1 - \alpha$, avec $Y \sim \chi^2(n - 1)$.
  
\end{solution}
\begin{solution}{5.7}
    On sait que
    \begin{align*}
      \mu &\in \bar{X} \pm z_{0,05} \frac{\sigma}{\sqrt{n}} \\
      &\in \bar{X} \pm 1,645 \frac{5}{\sqrt{n}}.
    \end{align*}
    La longueur de l'intervalle de confiance est $2 (1,645) (5) /
    \sqrt{n} = 16,45/\sqrt{n}$. Si l'on souhaite que $16,45/\sqrt{n}
    \leq 0,05$, alors $n \geq \nombre{108 241}$.
  
\end{solution}
\begin{solution}{5.8}
    On cherche à minimiser la longueur de l'intervalle de confiance
    $h(a, b) = (n-1) s^2 (b - a)/(ab)$ sous la contrainte que la
    probabilité dans cet intervalle est $1 - \alpha$, c'est-à-dire que
    $G(b) - G(a) = 1 - \alpha$. En utilisant la méthode des
    multiplicateurs de Lagrange, on pose
    \begin{displaymath}
      L(a, b, \lambda) = \frac{(n-1) s^2}{a} - \frac{(n-1) s^2}{b} +
      \lambda(G(b) - G(a) - 1 + \alpha).
    \end{displaymath}
    Les dérivées de cette fonction par rapport à chacune de ses
    variables sont:
    \begin{align*}
      \frac{\partial}{\partial a}\, L(a, b, \lambda)
      &= - \frac{(n-1) s^2}{a^2} + \lambda g(a) \\
      \frac{\partial}{\partial b}\, L(a, b, \lambda)
      &= - \frac{-(n-1) s^2}{b^2} + \lambda g(b) \\
      \frac{\partial}{\partial \lambda}\, L(a, b, \lambda)
      &= G(b) - G(a) - 1 + \alpha.
    \end{align*}
    En posant ces dérivées égales à zéro et en résolvant, on trouve
    que
    \begin{equation*}
      \frac{g(a)}{g(b)} = \frac{b^2}{a^2}
    \end{equation*}
    ou, de manière équivalente, que $b^2 g(b) = a^2 g(a)$.
  
\end{solution}
\begin{solution}{5.9}
\begin{enumerate}
\item
L'intervalle de confiance bilatéral pour $\mu$ a la forme
$$
\left[ \bar X_n - t_{n-1,\alpha/2} \frac{S_n}{\sqrt{n}}, \bar X_n + t_{n-1,\alpha/2} \frac{S_n}{\sqrt{n}} \right].
$$
Sachant que $n=20$ et $\alpha =0,1$, le quantile approprié de la distribution de Student est
$$
t_{19, \, 0,05} = 1.729.
$$
Avec $\bar x_n = 505$ et $S_n = 55$, l'intervalle devient
\begin{align*}
\left[ 505 -  1.729 \, \frac{55}{\sqrt{20}} \, , 505 + 1.729 \, \frac{55}{\sqrt{20}} \right] = [ 483.734 ,526.266].
\end{align*}

\item Puisque l'intervalle inclut $508$, il n'y a pas de preuve à l'effet que la moyenne du test de langue a significativement changé depuis 2005, à un niveau de confiance $90$~\%.

\item
L'intervalle a la même forme qu'en a) et le même quantile. Avec la moyenne échantillonnale et l'écart-type échantillonnal, l'intervalle peut être calculé comme suit
\begin{align*}
\left[ 495 -  1.729 \, \frac{70}{\sqrt{20}}\, ,495 + 1.729 \, \frac{70}{\sqrt{20}} \right] = [ 467.935, \,522.065].
\end{align*}
L'intervalle inclut $520$, il n'y a donc pas une différence significative dans la moyenne du test de mathématiques depuis 2005, à un niveau de confiance $90$~\%.

\item
Non, la méthode ne peut pas être utilisée car les échantillons ne sont pas indépendants, les tests ont été effectués par les 20 mêmes étudiants dans les deux cas.

\item
L'intervalle de confiance a la forme
$$
\left[\frac{n-1}{\chi^2_{n-1,\alpha/2}} \, S_n^2,\frac{n-1}{\chi^2_{n-1,1-\alpha/2}} \, S_n^2\right].
$$
Les quantiles appropriés d'une distribution khi-carrée avec $n-1=19$ degrés de liberté et $\alpha=0,1$ sont
$$
\chi^2_{n-1,1-\alpha/2} = 10.117, \quad \chi^2_{n-1,\alpha/2} = 30.144.
$$
Avec la variance échantillonnale $S_n^2 = 70^2 = 4900$, on obtient
\begin{align*}
\left[\frac{19}{30.144}\times4900,\frac{19}{10.117}\times4900\right] = [3088.557,9202.321].
\end{align*}
Une possibilité d'intervalle de confiance bilatéral à $100 \times (1-\alpha)$~\% pour $\sigma$ est
$$
\left[\sqrt{\frac{n-1}{\chi^2_{n-1,\alpha/2}}} \, S_n,\sqrt{\frac{n-1}{\chi^2_{n-1,1-\alpha/2}}} \, S_n\right]
$$
car
\begin{multline*}
\Pr\left(\sqrt{\frac{n-1}{\chi^2_{n-1,\alpha/2}}} \, S_n \le \sigma \le \sqrt{\frac{n-1}{\chi^2_{n-1,1-\alpha/2}}} \, S_n \right)\\ = \Pr\left(\frac{n-1}{\chi^2_{n-1,\alpha/2}} \, S_n^2 \le \sigma^2 \le \frac{n-1}{\chi^2_{n-1,1-\alpha/2}} \, S_n^2 \right) = 1-\alpha.
\end{multline*}
L'intervalle devient
\begin{align*}
[\sqrt{3088.557},\sqrt{9202.321}] = [55.575,95.929].
\end{align*}
\end{enumerate}
\end{solution}
\begin{solution}{5.10}
\begin{enumerate}
\item Les échantillons doivent être indépendants, normalement distribués et avoir la même variance $\sigma^2$.

\item L'estimation combinée de la variance est donnée par
$$
s^2 = \frac{(n-1) s_n^2 + (m-1)s_m^2}{n+m-2} = \frac{3\times 0,001 + 4 \times 0,002}{4+5-2} = 0.00157.
$$
Le quantile $97,5$\% de la loi de Student avec $n+m-2=7$ degrés de liberté est
$$
t_{0,025, \, 7} = 2.365.
$$
L'intervalle de confiance à 95~\% est donc donné par
\begin{multline*}
\left[\bar x_n - \bar y_m - t_{0,025, \, 7} s \sqrt{\frac{1}{n} + \frac{1}{m}} \, , \bar x_n - \bar y_m + t_{0,025, \, 7} s \sqrt{\frac{1}{n} + \frac{1}{m}}\right] \\ = \left[0,22 - 0,17 - t_{0,025, \, 7} s \sqrt{\frac{1}{4} + \frac{1}{5}}, 0,22 - 0,17 + t_{0,025, \, 7} s \sqrt{\frac{1}{4} + \frac{1}{5}}\right] \\= [-0.01288 \, ,0.11288].
\end{multline*}

\item Puisque $0$ est inclus dans l'intervalle de confiance bilatéral calculé en b), les moyennes ne semblent pas différer à un seuil de 5~\%.
\end{enumerate}
\end{solution}
\begin{solution}{5.11}
\begin{enumerate}
\item Les proportions échantillonnales sont
$$
p_n = \frac{126}{180}=0,7, \quad q_m = \frac{54}{100}=0,54.
$$
Les tailles d'échantillons sont respectivement de $n=180$ et $m=100$. Puisque les deux sont assez grandes, une approximation d'un intervalle de confiance à $90$~\% peut être utilisée pour $p-q$. On trouve
$$
\left[ p_n - q_m - z_{0,05} \sqrt{\frac{p_n(1-p_n)}{n} + \frac{q_m(1-q_m)}{m}},  p_n - q_m + z_{0,05} \sqrt{\frac{p_n(1-p_n)}{n} + \frac{q_m(1-q_m)}{m}}\right].
$$
Le quantile $95$~\% d'une loi $\mathcal{N}(0,1)$ est
$$
z_{0,05} = 1.645.
$$
L'intervalle de confiance devient
\begin{multline*}
\left [0.16 - 1.645\sqrt{\frac{0.7 \times 0.3}{180} + \frac{0.54\times 0.46}{100}} \, , 0.16 + 1.645\sqrt{\frac{0.7 \times 0.3}{180} + \frac{0.54\times 0.46}{100}}\right] \\ = [0.06062,0.25938].
\end{multline*}

\item Puisque l'intervalle de confiance calculé en a) ne contient pas $0$, la proportion d'enfants ainés semble être significativement plus grande dans la population d'étudiants gradués au niveau de confiance $90$~\%.

\item En supposant $n=m$, la taille d'échantillon de chaque groupe peut être calculée en résolvant l'équation
$$
1.645\sqrt{\frac{p(1-p)}{n} + \frac{q(1-q)}{n}} = 0,05.
$$
On trouve
$$
n = \left(\frac{1.645}{0.05}\right)^2 \{ p(1-p)+ q(1-q)\}.
$$
Sachant que pour tout $p\in(0,1)$, $p(1-p) \le 1/4$, une estimation conservatrice des tailles d'échantillon requises est
$$
n = m=  \left(\frac{1.645}{0.05}\right)^2 \times \left(\frac{1}{4} + \frac{1}{4} \right) \approx 541.205.
$$
Ainsi, pour atteindre la précision requise, chaque groupe doit comprendre au moins $542$ personnes.
\end{enumerate}
\end{solution}
\begin{solution}{5.12}
\begin{enumerate}
\item La vraisemblance est
$$
L(\lambda)=\prod_{i=1}^n \lambda e^{-\lambda x_i} = \lambda^n e^{-\lambda\sum_{i=1}^n x_i}
$$
et la log-vraisemblance est
$$
\ell(\lambda)=n\ln\lambda-\lambda\sum_{i=1}^n x_i.
$$
Si on dérive par rapport à $\lambda$, on trouve
$$
\ell'(\lambda)=\frac{n}{\lambda}-\sum_{i=1}^n x_i
$$
et $\ell''(\lambda)=-n/\lambda^2<0$, donc l'estimateur du maximum de vraisemblance est
$$
\frac{n}{\hat\lambda}-\sum_{i=1}^n x_i=0 \quad \Rightarrow \hat\lambda=\frac{1}{\bar x_n}.
$$

\item On a $\ln f(X;\lambda) = \ln\lambda-\lambda X$ donc l'information de Fisher est
\begin{align*}
I(\lambda)&=\ex\left[-\frac{\partial^2}{\partial\lambda^2} \ln f(X;\lambda) \right]=\ex\left[-\frac{\partial^2}{\partial\lambda^2} \left(\ln\lambda-\lambda X \right)\right]=\ex\left[\frac{1}{\lambda^2} \right] =\frac{1}{\lambda^2}.
\end{align*}

\item La distribution limite de l'EMV est
$$
\frac{\hat\lambda_n-\lambda}{\sqrt{1/\{nI(\lambda)\}}}
$$
est asymptotiquement $\mathcal{N}(0,1)$. On peut estimer $nI(\lambda)$ au dénominateur par $nI(\hat\lambda)=\frac{n}{\hat\lambda^2}$, pour obtenir, quand $n\to\infty$,
$$
\frac{\hat\lambda-\lambda}{\hat\lambda/\sqrt{n}}
$$
est asymptotiquement $\mathcal{N}(0,1)$. D'une autre façon, on peut estimer $nI(\lambda)$ par
$$
-\left.\frac{\partial^2}{\partial\lambda^2} \ell(\lambda)\right|_{\lambda=\hat\lambda}=\frac{n}{\hat\lambda^2},
$$
ce qui revient à la même réponse. Sachant que $z_{2,5\%}=1.96$ et $\hat\lambda=1/\bar x=1/105,2$, un intervalle de confiance approximatif de niveau 95~\% pour $\lambda$ est
\begin{align*}
\left[\hat\lambda-z_{\alpha/2}\sqrt{\frac{\hat\lambda^2}{n}},\hat\lambda+z_{\alpha/2}\sqrt{\frac{\hat\lambda^2}{n}}\right]&=\left[\frac{1}{\bar x_n}-z_{\alpha/2}\frac{1}{\sqrt{n}\bar x_n}, \frac{1}{\bar x_n}+z_{\alpha/2}\frac{1}{\sqrt{n}\bar x_n}\right]\\
&=\left[\frac{1}{105.2}-\frac{1.96}{10\times105.2},\frac{1}{105.2}+\frac{1.96}{10\times105.2}\right]\\
&=\left[0.00764,0.01137\right].
\end{align*}

\item Le paramètre d'intérêt est $\theta=\Pr[X>300]=e^{-300\lambda}$. On a
\begin{align*}
\Pr\left[\frac{1}{\bar X_n}-z_{\alpha/2}\frac{1}{\sqrt{n}\bar X_n}\leq \lambda \leq \frac{1}{\bar X_n}+z_{\alpha/2}\frac{1}{\sqrt{n}\bar X_n}\right]\approx 0,95.
\end{align*}
Multiplier l'inégalité par $-300$ change le signe d'inégalité, alors que l'exponentielle est une fonction croissante, donc
\begin{align*}
\Pr\left[\exp\left\{-300\left(\frac{1}{\bar X_n}-z_{\alpha/2}\frac{1}{\sqrt{n}\bar X_n}\right)\right\}\geq e^{-300\lambda} \geq \exp\left\{-300\left(\frac{1}{\bar X_n}+z_{\alpha/2}\frac{1}{\sqrt{n}\bar X_n}\right)\right\}\right]\approx 0,95.
\end{align*}
Donc, un intervalle de confiance de niveau approximatif $95$~\% pour $\Pr[X>300]$ est
$$
\left[0.03302,0.10099\right].
$$
\end{enumerate}
\end{solution}
